\documentclass{dgovdoc}

\usepackage[swedish]{babel}
\usepackage[T1]{fontenc}

\usepackage{hyperref}

\title{Verksamhetsplan 2011}

\begin{document}

\maketitle

\section{D-rektoratet}

\subsection{Kårobligatoriet}

D-rektoratet skall

\begin{itemize}
\item synliggöra kårens och sektionens arbete i samarbete med Informationsorganet.
\item öka antalet medlemmar för att behålla sektionens trovärdighet gentemot
  skolan, målet är att över 50\% av de studerande vid datateknikprogrammet på CSC
  skall vara medlemmar till årsskiftet 2011/2012.
\end{itemize}

\subsection{Löpande arbete}

Under det gångna året har D-rektoratet haft en rutin att träffas en gång i
veckan för gemensam lunch och avstämning av det löpande arbetet. Det har
fungerat mycket bra och fått arbetet att flyta framåt.

D-rektoratet skall

\begin{itemize}
\item minst en gång i månaden genomföra arbetsmöten eller Kreativa forum där
  handfast styrelsearbete sker gemensamt.
\item en gång i veckan träffas under lunchen och diskutera styrelsefrågor.
\item lämna återkoppling på frågor och funderingar från medlemmarna inom en
  arbetsdag.
\item minst en gång i månaden ha ett D-rektoratsmöte.
\end{itemize}

\subsection{Ekonomi}

D-rektoratet skall

\begin{itemize}
\item mötas minst en gång i månaden och tillsammans med kassören sköta den
  löpande bokföringen.
\item utreda konsekvenserna av den nya momslagstiftningen som väntas träda i
  kraft 1 Januari 2012 och genomföra nödvändiga åtgärder för att sektionen
  ska följa den nya lagstiftningen.
\end{itemize}

\subsection{Andra sektioner och systerprogram}

Under det gångna året har sektionen fortsatt sitt nära samarbete med Sektionen
för Medieteknik, ett samarbete kallat Team CSC. Inom ramen för detta samarbete
har styrelserna från respektive sektioner drivit gemensamma frågor gentemot
skolan och inom THS. Detta samarbete har gett gott resultat i de frågor som
drivits.

D-rektoratet skall

\begin{itemize}
\item arbeta för att behålla och vidareutveckla Team CSC.
\item aktivt arbeta för att hitta frågor som kan drivas gemensamt av flera
  sektioner och ta initiativ till sådant samarbete över sektionsgränserna.
\item verka för att ha ett nära samarbete med andra sektioners styrelser.
\item besöka andra sektioners styrelser och diskutera aktuella händelser på
  KTH. Detta görs med fördel under avslappnade former.
\end{itemize}

\subsection{Ökat sektionsengagemang}

D-rektoratet skall

\begin{itemize}
\item finna tillfällen för nämnderna att synas även efter mottagningen.
\end{itemize}

\subsection{Övrigt}

D-rektoratet skall

\begin{itemize}
\item anordna minst fyra D-funkmiddagar för att öka kommunikation mellan
  funktionärer och nämnder.
\item anordna två teambuilding-events för Datasektionens funktionärer i syfte
  att öka sammanhållningen mellan nämnderna.
\item anordna en Skiftesgasque för att tacka sektionsaktiva medlemmar.
\item anordna två styrelseöverlämningar, en efter varje SM som enligt stadgarna
  ska behandla val av styrelseledamöter.
\item tillse att DKMs lagerhållningssystem byts ut mot ett system som både
  klarar lagerhållning och dagliga kreditköp.
\end{itemize}

\section{Nämnder}

\subsection{METAdorerna}

METAdorerna skall

\begin{itemize}
\item regelbundet genomföra underhåll av de maskiner och delar av
  sektionslokalen som kräver löpande underhåll och tillsyn.
\item regelbundet kontrollera att alla inventarier i sektionslokalen är hela
  och rena samt ersätta, reparera eller ta bort trasiga inventarier.
\item se till att sektionslokalen städas regelbundet.
\item anordna en tackfest en gång per termin för dem som hjälpt till att städa
  sektionslokalen.
\item ha ett samarbete med motsvarande nämnd på Sektionen för Medieteknik.
\end{itemize}

\subsection{Idrottsnämnden}

Idrottsnämnden skall

\begin{itemize}
\item anordna regelbundna träningstillfällen för sektionens medlemmar.
\end{itemize}

\subsection{Informationsorganet}

Ior skall

\begin{itemize}
\item hjälpa D-rektoratet och övriga nämnder med informationsspridning.
\item underhålla och vidareutveckla sektionens webbplats.
\item ansvara för drift och underhåll av sektionens datorresurser.
\item ha en samordnande roll för sektionens informationsflöden.
\item ha regelbunden kontakt med sektionens nämnder i syfte att höja kvalitén
  på informationsspridningen.
\end{itemize}

\subsection{Internationella utskottet}

DIU skall

\begin{itemize}
\item samordna sektionens internationella verksamhet. Detta inkluderar att
  hålla kontakten med ISS ordförande på THS och ansvariga för
  utbytesstudier på CSC:s kansli och institutionerna.
\item hålla sektionsmedlemmarna informerade om den internationella verksamheten
  på sektionen.
\item hjälpa till i THS centrala mottagningsverksamhet för utländska studenter.
  Detta inkluderar att rekrytera faddrar och koordinera fadderverksamheten.
\item fungera som kontaktyta för utländska studenter på sektionen.
\end{itemize}

\subsection{Jämlikhetsnämnden}

Jämlikhetsnämnden skall

\begin{itemize}
\item genomföra minst en undersökning om hur jämlikhetssituationen på sektionen
  är.
\item utvärdera resultaten av undersökningen och presentera dessa för
  D-rektoratet och sektionens medlemmar.
\item upprätta en tydlig kontakt dit medlemmar kan vända sig med
  jämlikhetsfrågor.
\item verka för att utöka ett samarbete mellan jämlikhetsnämnden och
  jämlikhetsgruppen på CSC.
\end{itemize}

\subsection{Klubbmästeriet}

DKM skall

\begin{itemize}
\item arrangera onsdagspubar.
\item arrangera Plums.
\item arrangera minst fyra fester för sektionens medlemmar.
\item bokföra sin verksamhet regelbundet.
\item anordna tentapub efter varje tentaperiod
\end{itemize}

\subsection{Mottagningen}

Mottagningen skall

\begin{itemize}
\item hålla en mottagning för de nya studenterna som börjar till hösten med
  syftet att få dem att känna sig välkomna och introducera dem till KTH,
  THS och sektionen.
\item eftersträva att få n0llan att vilja bli medlemmar i sektionen och THS.
\item informera n0llan om THS centralts funktion och verksamhet.
\item fungera som en naturlig inkörsport till sektionsengagemang och socialt
  umgänge med andra studenter.
\item ta in feedback från nØllan och sammanställa informationen skriftligt i
  syftet att förbättra mottagningen från nØllans synvinkel.
\end{itemize}

\subsection{Studienämnden}

Studienämndens skall

\begin{itemize}
\item regelbundet begära ut de kursvärderingar som skolan enligt
  högskoleförordningen är skyldig att göra. Dessa ska användas som
  diskussionsunderlag på studienämndens möten och följas upp.
\item arbeta för att det ska utses minst två årskursrepresentanter från årskurs
  ett till tre. Det bör också finnas en representant från varje masterprogram.
\item regelbundet lägga upp renskrivna mötesanteckningar på hemsidan.
\item uppdatera sin hemsida med relevant information så att den på ett bra sätt
  representerar studienämndens syfte och arbete samt att det finns aktuell
  information för studenterna.
\end{itemize}

\subsection{Näringslivsgruppen}

Näringslivsgruppen skall

\begin{itemize}
\item anordna D-Dagen.
\item ge företag annonsmöjligheter för event och jobb
\item sammanställa ett dokument med rutiner i samband med
  företagspresentationer på campus.
\item ansvara för att Konglig Datasektionens varumärke upprätthålls mot
  näringslivet.
\item löpande underhålla en databas över sektionens företagskontakter.
\item upprätta rutiner för uppföljning av sina egna fakturor.
\item se över och följa upp eventuella sponsorkontrakt rörande sektionen.
\item skriva en policy för sektionens företagskontaker och sponsrade event.
\item öka kommunikationen mellan lärare på KTH om möjligheterna för informativa
  företagsföreläsningar.
\item öka näringslivsgruppens antal aktiva medlemmar till minst tio stycken.
\item stärka näringslivsgruppen som grupp genom teambuilding och egna interna
  event.
\end{itemize}

\subsection{Qulturnämnden}

Qulturnämnden skall

\begin{itemize}
\item anordna regelbundna aktiviteter för sektionsmedlemmarna, såsom
  filmvisningar, spelkvällar och liknande.
\item tillhandahålla böcker, spel och annat kulturellt i sektionslokalen
\item anordna arrangemang utanför sektionslokalen, såsom besök på
  utställningar, teater eller annan kulturellt relaterad verksamhet.
\end{itemize}

\subsection{Redaqtionen}

Redaqtionen skall

\begin{itemize}
\item ge ut en nØlledBuggen under mottagningen.
\item träffas regelbundet för att arbeta fram en ny dBuggen.
\item ge ut minst en dBuggen utöver nØlledBuggen.
\end{itemize}

\subsection{Spexmästeriet}

Spexmästeriet skall

\begin{itemize}
\item arbeta för att skriva och uppföra ett spex.
\item ha regelbundna möten för att främja arbetet med spexet.
\item sprida information om Spexmästeriet för att locka fler att deltagare.
\end{itemize}

\end{document}
